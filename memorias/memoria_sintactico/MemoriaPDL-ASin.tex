\documentclass[12pt,twoside,letterpaper]{article}
%NOTE: This report format is 
\setlength{\footskip}{1.5cm}

% report information

\newcommand{\reporttitle}{Procesadores de Lenguajes: Memoria del Proyecto}
\newcommand{\reportauthorOne}{Jose Luis Prado Sierra }
\newcommand{\cidOne}{220070}
\newcommand{\reportauthorTwo}{Alejandro Gragera Serradilla }
\newcommand{\cidTwo}{22M043}
\newcommand{\reportauthorThree}{Antonio Bielza Díez }
\newcommand{\cidThree}{22M049}
\newcommand{\reporttype}{Coursework}

% include files that load packages and define macros
\input{includes} % various packages needed for maths etc.
\input{notation} % short-hand notation and macros


%%%%%%%%%%%%%%%%%%%%%%%%%%%%

\raggedbottom
\begin{document}
% front page
% Last modification: 2016-09-29 (Marc Deisenroth)
% Modification for UW: 2017-05-22 (jphickey)
\begin{titlepage}

\newcommand{\HRule}{\rule{\linewidth}{0.5mm}} % Defines a new command for the horizontal lines, change thickness here


%----------------------------------------------------------------------------------------
%	LOGO SECTION
%----------------------------------------------------------------------------------------



\begin{center} % Center remainder of the page

%----------------------------------------------------------------------------------------
%	HEADING SECTIONS
%----------------------------------------------------------------------------------------

\includegraphics[width = 9cm]{./figures/etsiinf}\\[1.5cm] 
%\textbf{\textsc{\Large Procesadores de Lenguajes}}\\[1.0cm] 
\textsc{\Large Universidad Politécnica de Madrid}\\[0.5cm] 

%----------------------------------------------------------------------------------------
%	TITLE SECTION
%----------------------------------------------------------------------------------------
\vspace{0.75cm}
\HRule \\[0.4cm]
{ \huge \bfseries \reporttitle}\\ % Title of your document
\HRule \\[0.7cm]
    \textsc{\large Analizador Léxico y Tabla de Símbolos}
\end{center}
%----------------------------------------------------------------------------------------
%	AUTHOR SECTION
%----------------------------------------------------------------------------------------

%\begin{minipage}{0.4\hsize}
\vfill
\begin{flushright} \large
    \textsc{\textbf{Grupo 18}}
\\
\reportauthorOne - \cidOne\\ % Your name
\reportauthorTwo - \cidTwo\\ % Your name
\reportauthorThree - \cidThree\\ % Your name
\end{flushright}
%\vspace{4cm}
%\makeatletter
%Date: \@date 

%\vfill % Fill the rest of the page with whitespace



\makeatother


\end{titlepage}




%%%%%%%%%%%%%%%%%%%%%%%%%%% table of content
%If a table of content is needed, simply uncomment the following lines
\tableofcontents
\newpage

%%%%%%%%%%%%%%%%%%%%%%%%%%%% Main document
%\section*{Note:}
%\emph{This document is intended to provide a sample structure for the reports in ME303 at the University of Waterloo. }

\section{Analizador Sintáctico}

Durante el desarrollo del Analizador Sintáctico hemos descrito una gramática y desarrollado un programa que, junto al analizador léxico, es capaz de reconocer y analizar correctamente el lenguaje planteado para el desarrollo de la práctica.

\subsection{Gramática}

La gramática planteada para el desarrollo de Analizador Sintáctico es de tipo 2 segun la jerarquía de Chomsky, es decir, una gramática libre de contexto. Además cumple con las propiedadees necesarias para ser LL(1), lo que nos permite desarrollar un analizador sintáctico predictivo.
\\\\
Teniendo todo lo anterior en cuenta, la gramática definida es la siguiente:
\begin{enumerate}[label=\textbf{\arabic*:}]
  \item \texttt{S} $\to$ \texttt{LC S | LF S | eof}
  \item \texttt{LC} $\to$ \texttt{LS semicolon | if oppar Expresion clpar CuerpoIf | else CuerpoIf}
  \item \texttt{LF} $\to$ \texttt{function TypeFun id oppar Args clpar opbra Cuerpo clbra}
  \item \texttt{CuerpoIf} $\to$ \texttt{opbra Cuerpo clbra | LC}
  \item \texttt{Cuerpo} $\to$ \texttt{LC Cuerpo |} $\lambda$
  \item \texttt{Args} $\to$ \texttt{Tipo id ArgMore | void}
  \item \texttt{ArgsLlamada} $\to$ \texttt{Expresion ArgMoreLlamada |} $\lambda$
  \item \texttt{ArgMoreLlamada} $\to$ \texttt{comma Expresion ArgMoreLlamada |} $\lambda$
  \item \texttt{ArgMore} $\to$ \texttt{comma Tipo id ArgMore |} $\lambda$
  \item \texttt{LS} $\to$ \texttt{let Tipo id Asignar | id IdOpt | read id | write Expresion | return ExpReturn}
  \item \texttt{IdOpt} $\to$ \texttt{oppar ArgsLlamada clpar | eq Expresion | pluseq Expresion}
  \item \texttt{TypeFun} $\to$ \texttt{void | Tipo}
  \item \texttt{Tipo} $\to$ \texttt{int | float | string | boolean}
  \item \texttt{Asignar} $\to$ \texttt{eq Expresion |} $\lambda$
  \item \texttt{ExpReturn} $\to$ \texttt{Expresion |} $\lambda$
  \item \texttt{Expresion} $\to$ \texttt{Expresion1 ExpresionAux}
  \item \texttt{ExpresionAux} $\to$ \texttt{and Expresion |} $\lambda$
  \item \texttt{Expresion1} $\to$ \texttt{Expresion2 Expresion1Aux}
  \item \texttt{Expresion1Aux} $\to$ \texttt{minorthan Expresion1 |} $\lambda$
  \item \texttt{Expresion2} $\to$ \texttt{Expresion3 Expresion2Aux}
  \item \texttt{Expresion2Aux} $\to$ \texttt{sum Expresion2 |} $\lambda$
  \item \texttt{Expresion3} $\to$ \texttt{oppar Expresion clpar | intconst | realconst | string | true | false | id Expresion4}
  \item \texttt{Expresion4} $\to$ \texttt{oppar ArgsLlamada clpar |} $\lambda$
\end{enumerate}

\subsection{Demostración de que la gramática es LL(1)}


Para demostrar que una gramática es LL(1), debemos verificar que cumple las siguientes propiedades:

\begin{enumerate}[itemsep=0.2em, topsep=0.5em]
    \item No existe recursividad por la izquierda.
    \item Está factorizada por la izquierda.
    \item Para cada no terminal $A$ con producciones alternativas $A \rightarrow \alpha_1 ~|~ \alpha_2 ~|~ \ldots ~|~ \alpha_n$, los conjuntos FIRST($\alpha_{i}$) son disjuntos.
    \item Para cada producción con $\lambda$: $A \rightarrow \alpha ~|~ \lambda$, se cumple que $\text{FIRST}(\alpha) \cap \text{FOLLOW}(A) = \emptyset$.
\end{enumerate}

\subsubsection{Cálculo de Conjuntos FIRST}

El conjunto $\text{FIRST}(\alpha)$ contiene todos los terminales que pueden aparecer al inicio de una derivación desde $\alpha$.

\vspace{0.3em}
\textbf{Conjuntos FIRST principales:}
\begin{align*}
\text{FIRST}(Tipo) ={}& \{\mathit{int}, \mathit{float}, \mathit{string}, \mathit{boolean}\} \\
\text{FIRST}(TypeFun) ={}& \{\mathit{void}, \mathit{int}, \mathit{float}, \mathit{string}, \mathit{boolean}\} \\
\text{FIRST}(LS) ={}& \{\mathit{let}, \mathit{id}, \mathit{read}, \mathit{write}, \mathit{return}\} \\
\text{FIRST}(LC) ={}& \{\mathit{let}, \mathit{id}, \mathit{read}, \mathit{write}, \mathit{return}, \mathit{if}, \mathit{else}\} \\
\text{FIRST}(LF) ={}& \{\mathit{function}\} \\
\text{FIRST}(S) ={}& \{\mathit{let}, \mathit{id}, \mathit{read}, \mathit{write}, \mathit{return}, \mathit{if}, \mathit{else}, \mathit{function}, \mathit{eof}\} \\
\text{FIRST}(Expresion3) ={}& \{\mathit{oppar}, \mathit{intconst}, \mathit{realconst}, \mathit{string}, \mathit{true}, \mathit{false}, \mathit{id}\} \\
\text{FIRST}(Expresion2) ={}& \text{FIRST}(Expresion3) \\
\text{FIRST}(Expresion1) ={}& \text{FIRST}(Expresion2) \\
\text{FIRST}(Expresion) ={}& \text{FIRST}(Expresion1) \\
\text{FIRST}(Args) ={}& \{\mathit{int}, \mathit{float}, \mathit{string}, \mathit{boolean}, \mathit{void}\} \\
\text{FIRST}(ArgsLlamada) ={}& \{\mathit{oppar}, \mathit{intconst}, \mathit{realconst}, \mathit{string}, \mathit{true}, \mathit{false}, \mathit{id}, \lambda\} \\
\text{FIRST}(Cuerpo) ={}& \{\mathit{let}, \mathit{id}, \mathit{read}, \mathit{write}, \mathit{return}, \mathit{if}, \mathit{else}, \lambda\} \\
\text{FIRST}(CuerpoIf) ={}& \{\mathit{opbra}, \mathit{let}, \mathit{id}, \mathit{read}, \mathit{write}, \mathit{return}, \mathit{if}, \mathit{else}\}
\end{align*}

\subsubsection{Cálculo de Conjuntos FOLLOW}

El conjunto $\text{FOLLOW}(A)$ contiene todos los terminales que pueden aparecer inmediatamente después de $A$ en alguna derivación.

\vspace{0.3em}
\textbf{Conjuntos FOLLOW principales:}
\begin{align*}
\text{FOLLOW}(S) ={}& \{\mathit{eof}\} \\
\text{FOLLOW}(Cuerpo) ={}& \{\mathit{clbra}\} \\
\text{FOLLOW}(LS) ={}& \{\mathit{semicolon}\} \\
\text{FOLLOW}(Args) ={}& \{\mathit{clpar}\} \\
\text{FOLLOW}(ArgsLlamada) ={}& \{\mathit{clpar}\} \\
\text{FOLLOW}(LC) ={}& \{\mathit{let}, \mathit{id}, \mathit{read}, \mathit{write}, \mathit{return}, \mathit{if}, \mathit{else}, \mathit{function}, \mathit{eof}, \mathit{clbra}\} \\
\text{FOLLOW}(LF) ={}& \{\mathit{let}, \mathit{id}, \mathit{read}, \mathit{write}, \mathit{return}, \mathit{if}, \mathit{else}, \mathit{function}, \mathit{eof}\} \\
\text{FOLLOW}(CuerpoIf) ={}& \{\mathit{let}, \mathit{id}, \mathit{read}, \mathit{write}, \mathit{return}, \mathit{if}, \mathit{else}, \mathit{function}, \mathit{eof}, \mathit{clbra}\} \\
\text{FOLLOW}(Expresion) ={}& \{\mathit{clpar}, \mathit{semicolon}, \mathit{comma}, \mathit{and}, \mathit{minorthan}, \mathit{sum}\}
\end{align*}

\subsubsection{Verificación de la Propiedad LL(1)}

Debemos verificar que para cada no terminal con producciones alternativas, los conjuntos FIRST son disjuntos, y para producciones con $\lambda$, se cumple que $\text{FIRST}(\alpha) \cap \text{FOLLOW}(A) = \emptyset$.

\vspace{0.3em}
\textbf{Caso 1: No terminal $S$}

\begin{center}
\begin{minipage}{0.85\textwidth}
\begin{align*}
S \rightarrow{}& LC~S ~|~ LF~S ~|~ \mathit{eof} \\
\text{FIRST}(LC~S) ={}& \{\mathit{let}, \mathit{id}, \mathit{read}, \mathit{write}, \mathit{return}, \mathit{if}, \mathit{else}\} \\
\text{FIRST}(LF~S) ={}& \{\mathit{function}\} \\
\text{FIRST}(\mathit{eof}) ={}& \{\mathit{eof}\} \quad \text{Los tres conjuntos son disjuntos: } \checkmark
\end{align*}
\end{minipage}
\end{center}


\textbf{Caso 2: No terminal $LC$}
\begin{align*}
LC \rightarrow{}& LS~; ~|~ \mathit{if}~(~\ldots ~|~ \mathit{else}~\ldots \\
\text{FIRST}(LS~;) ={}& \{\mathit{let}, \mathit{id}, \mathit{read}, \mathit{write}, \mathit{return}\} \\
\text{FIRST}(\mathit{if}\ldots) ={}& \{\mathit{if}\} \\
\text{FIRST}(\mathit{else}\ldots) ={}& \{\mathit{else}\} \quad \text{Los tres conjuntos son disjuntos: } \checkmark
\end{align*}

\textbf{Caso 3: No terminal $LS$ (5 producciones)}
\begin{align*}
LS \rightarrow{}& \mathit{let}~\ldots ~|~ \mathit{id}~\ldots ~|~ \mathit{read}~\ldots ~|~ \mathit{write}~\ldots ~|~ \mathit{return}~\ldots \\
\text{FIRST}(\mathit{let}\ldots) ={}& \{\mathit{let}\}, \quad \text{FIRST}(\mathit{id}\ldots) = \{\mathit{id}\}, \quad \text{FIRST}(\mathit{read}\ldots) = \{\mathit{read}\} \\
\text{FIRST}(\mathit{write}\ldots) ={}& \{\mathit{write}\}, \quad \text{FIRST}(\mathit{return}\ldots) = \{\mathit{return}\} \quad \text{Todos disjuntos: } \checkmark
\end{align*}

\textbf{Caso 4: No terminal $Cuerpo$ (con $\lambda$)}
\begin{align*}
Cuerpo \rightarrow{}& LC~Cuerpo ~|~ \lambda \\
\text{FIRST}(LC~Cuerpo) ={}& \{\mathit{let}, \mathit{id}, \mathit{read}, \mathit{write}, \mathit{return}, \mathit{if}, \mathit{else}\} \\
\text{FOLLOW}(Cuerpo) ={}& \{\mathit{clbra}\} \quad \text{Verificación: } \text{FIRST}(LC~Cuerpo) \cap \text{FOLLOW}(Cuerpo) = \emptyset \quad \checkmark
\end{align*}

\textbf{Caso 5: No terminal $ArgsLlamada$ (con $\lambda$)}
\begin{align*}
ArgsLlamada \rightarrow{}& Expresion~ArgMoreLlamada ~|~ \lambda \\
\text{FIRST}(Expresion\ldots) ={}& \{\mathit{oppar}, \mathit{intconst}, \mathit{realconst}, \mathit{string}, \mathit{true}, \mathit{false}, \mathit{id}\} \\
\text{FOLLOW}(ArgsLlamada) ={}& \{\mathit{clpar}\} \quad \text{Verificación: } \text{FIRST}(Expresion\ldots) \cap \text{FOLLOW}(ArgsLlamada) = \emptyset \quad \checkmark
\end{align*}

\textbf{Casos 6 y 7: No terminales $ArgMore$ y $ArgMoreLlamada$ (con $\lambda$)}
\begin{align*}
ArgMore \rightarrow{}& ,~Tipo~id~ArgMore ~|~ \lambda \\
\text{FIRST}(,\ldots) ={}& \{\mathit{comma}\} \quad \text{FOLLOW}(ArgMore) = \{\mathit{clpar}\} \quad \checkmark \\
ArgMoreLlamada \rightarrow{}& ,~Expresion~ArgMoreLlamada ~|~ \lambda \\
\text{FIRST}(,\ldots) ={}& \{\mathit{comma}\} \quad \text{FOLLOW}(ArgMoreLlamada) = \{\mathit{clpar}\} \quad \checkmark
\end{align*}

\textbf{Casos 8 y 9: No terminales $Asignar$ y $ExpReturn$ (con $\lambda$)}
\begin{align*}
Asignar \rightarrow{}& =~Expresion ~|~ \lambda \\
\text{FIRST}(=\ldots) ={}& \{\mathit{eq}\} \quad \text{FOLLOW}(Asignar) = \{\mathit{semicolon}\} \quad \checkmark \\
ExpReturn \rightarrow{}& Expresion ~|~ \lambda \\
\text{FIRST}(Expresion) ={}& \{\mathit{oppar}, \mathit{intconst}, \mathit{realconst}, \mathit{string}, \mathit{true}, \mathit{false}, \mathit{id}\} \\
\text{FOLLOW}(ExpReturn) ={}& \{\mathit{semicolon}\} \quad \text{Verificación: } \text{FIRST}(Expresion) \cap \{\mathit{semicolon}\} = \emptyset \quad \checkmark
\end{align*}

\textbf{Caso 10: No terminales de expresiones (con $\lambda$)}
\begin{align*}
ExpresionAux \rightarrow{}& \mathit{and}~Expresion ~|~ \lambda \\
\text{FIRST}(\mathit{and}\ldots) ={}& \{\mathit{and}\} \quad \text{FOLLOW}(ExpresionAux) = \{\mathit{clpar}, \mathit{semicolon}, \mathit{comma}\} \quad \checkmark \\
Expresion1Aux \rightarrow{}& \mathit{minorthan}~Expresion1 ~|~ \lambda \\
\text{FIRST}(\mathit{minorthan}\ldots) ={}& \{\mathit{minorthan}\} \quad \text{FOLLOW}(Expresion1Aux) = \{\mathit{clpar}, \mathit{semicolon}, \mathit{comma}, \mathit{and}\} \quad \checkmark \\
Expresion2Aux \rightarrow{}& \mathit{sum}~Expresion2 ~|~ \lambda \\
\text{FIRST}(\mathit{sum}\ldots) ={}& \{\mathit{sum}\} \quad \text{FOLLOW}(Expresion2Aux) = \{\mathit{clpar}, \mathit{semicolon}, \mathit{comma}, \mathit{and}, \mathit{minorthan}\} \quad \checkmark \\
Expresion4 \rightarrow{}& (~ArgsLlamada~) ~|~ \lambda \\
\text{FIRST}((\ldots) ={}& \{\mathit{oppar}\} \quad \text{FOLLOW}(Expresion4) = \{\mathit{clpar}, \mathit{semicolon}, \mathit{comma}, \mathit{and}, \mathit{minorthan}, \mathit{sum}\} \quad \checkmark
\end{align*}

\begin{flushright}
$\square$
\end{flushright}

\subsection{Tabla LL y Pseudocódigo}
    Se ha hecho un renombrado de los símbolos de la gramática para favorecer su representación.
    \\\\
    CuerpoIf $\to$ CI \\
    Cuerpo $\to$ CU \\
    Args $\to$ AR \\
    ArgsLlamada $\to$ AL \\
    ArgMore $\to$ AM \\
    ArgMoreLlamada $\to$ AML \\
    IdOpt $\to$ IO \\
    TypeFun $\to$ TF \\
    Tipo $\to$ TI \\
    Asignar $\to$ AS \\
    ExpReturn $\to$ ER \\
    Expresion $\to$ EX \\
    ExpresionAux $\to$ EA \\
    Expresion1 $\to$ E1 \\
    Expresion1Aux $\to$ E1A \\
    Expresion2 $\to$ E2 \\
    Expresion2Aux $\to$ E2A \\
    Expresion3 $\to$ E3 \\
    Expresion4 $\to$ E4 \\

    La tabla por gestión de espacio esta representada en la página \hyperref[sec:tabla]{\pageref*{sec:tabla}} dentro del Anexo.

\subsection{Errores}

Un analizador sintáctico ha de ser capaz de detectar los posibles errores en cada línea y notificarlos de forma clara y localizada al usuario.
\\
Para esta entrega hemos detectado los siguientes casos de error:
\begin{itemize}[itemsep=0.0em, topsep=0.5em]
    \item \textbf{Línea mal terminada:} El analizador detecta que una línea no ha terminado con punto y coma cuando debería.
    \item \textbf{Variable mal declarada:} Cuando a la declaración de una variable le falta alguno de sus componentes: let, tipo...
    \item \textbf{Función mal declarada:} Cuando a la declaración de una función le falta alguno de sus componentes: paréntesis, tipos, comas, corchetes...
    \item \textbf{Función mal llamada:} Cuando se llama a una función de forma incorrecta: sin paréntesis, agregando tipos a las variables...
    \item \textbf{Sentencia if mal declarada:} Cuando una sentencia if no es correcta: multiple línea sin usar corchetes, falta de paréntesis...
    \item \textbf{Sentencia else mal declarada:} Cuando una sentencia else no es correcta, igual que con el if.
    \item \textbf{Expresión mal delcarada:} Cuando la declaración de una expresión de cualquier tipo no es correcta, por ejemplo: (1+2*3 o num1 $<$ ().
\end{itemize}

\newpage
\section{Anexo}

\subsection{Casos Correctos}

\subsubsection{Caso Correcto}
\begin{itemize}[label={-}, itemsep=-1em, topsep=0.5em, parsep=-0.2em]
    \item \textbf{Código:}
    \begin{verbatim} 
let int a_53f = 3;
let boolean bb;
let float fl = 3.5;
let int c;
let float fl2;
c = a_53f + 9;
fl2 = fl + 3.0;
let string str = 'Supercalifragilisticoespialidoso';
    \end{verbatim}
    \item \textbf{Archivo de Tokens:}\begin{verbatim}
<LET,>
<INT,>
<ID,0>
<EQ,>
<INTCONST,3>
<SEMICOLON,>
<LET,>
<BOOLEAN,>
<ID,1>
<SEMICOLON,>
<LET,>
<FLOAT,>
<ID,2>
<EQ,>
<REALCONST,3.5>
<SEMICOLON,>
<LET,>
<INT,>
<ID,3>
<SEMICOLON,>
<LET,>
<FLOAT,>
<ID,4>
<SEMICOLON,>
<ID,5>
<EQ,>
<ID,6>
<SUM,>
<INTCONST,9>
<SEMICOLON,>
<ID,7>
<EQ,>
<ID,8>
<SUM,>
<REALCONST,3.0>
<SEMICOLON,>
<LET,>
<STRING,>
<ID,9>
<EQ,>
<STR,"Supercalifragilisticoespialidoso">
<SEMICOLON,>
<EOF,>
    \end{verbatim}
    \item \textbf{Archivo de Tabla de Símbolos:}\begin{verbatim}
CONTENIDOS DE LA TABLA # 1:
* LEXEMA : 'a_53f'
  Atributos:
--------- ---------
* LEXEMA : 'bb'
  Atributos:
--------- ---------
* LEXEMA : 'fl'
  Atributos:
--------- ---------
* LEXEMA : 'c'
  Atributos:
--------- ---------
* LEXEMA : 'fl2'
  Atributos:
--------- ---------
* LEXEMA : 'str'
  Atributos:
--------- ---------
    \end{verbatim}
\end{itemize}

\subsubsection{Caso Correcto}
\begin{itemize}[label={-}, itemsep=-1em, topsep=0.5em, parsep=-0.2em]
    \item \textbf{Código:}
    \begin{verbatim} 
let int s = 2; // Esto es una prueba de comentario 4hjdsh2&&ndj!
s = s + 3; // Sigue siendo una prueba jdlwn;eldr4,4nc,erioe4nx,(·!N$"()E·/$"N
let boolean bs = s && 4;
    \end{verbatim}
    \item \textbf{Archivo de Tokens:}
    \begin{verbatim} 
<LET,>
<INT,>
<ID,0>
<EQ,>
<INTCONST,2>
<SEMICOLON,>
<ID,1>
<EQ,>
<ID,2>
<SUM,>
<INTCONST,3>
<SEMICOLON,>
<LET,>
<BOOLEAN,>
<ID,3>
<EQ,>
<ID,4>
<AND,>
<INTCONST,4>
<SEMICOLON,>
<EOF,>
    \end{verbatim}
    \item \textbf{Archivo de Tabla de Símbolos:}
    \begin{verbatim} 
CONTENIDOS DE LA TABLA # 1:
* LEXEMA : 's'
  Atributos:
--------- ---------
* LEXEMA : 'bs'
  Atributos:
--------- ---------
    \end{verbatim}
\end{itemize}

\newpage
\subsubsection{Caso Correcto}
\begin{itemize}[label={-}, itemsep=-1em, topsep=0.5em, parsep=-0.2em]
    \item \textbf{Código:}
    \begin{verbatim} 
function void sql {
	lt boolean b = true;
		let int dl = 7;
	if (b == true) {
		dl = dl + 2;
		write('Sí');
	}
	else return;
}
    \end{verbatim}
    \item \textbf{Archivo de Tokens:}
    \begin{verbatim} 
<FUNCTION,>
<VOID,>
<ID,0>
<OPBRA,>
<ID,1>
<BOOLEAN,>
<ID,2>
<EQ,>
<TRUE,>
<SEMICOLON,>
<LET,>
<INT,>
<ID,3>
<EQ,>
<INTCONST,7>
<SEMICOLON,>
<IF,>
<OPPAR,>
<ID,4>
<EQ,>
<EQ,>
<TRUE,>
<CLPAR,>
<OPBRA,>
<ID,5>
<EQ,>
<ID,6>
<SUM,>
<INTCONST,2>
<SEMICOLON,>
<WRITE,>
<OPPAR,>
<STR,"Sí">
<CLPAR,>
<SEMICOLON,>
<CLBRA,>
<ELSE,>
<RETURN,>
<SEMICOLON,>
<CLBRA,>
<EOF,>
    \end{verbatim}
    \item \textbf{Archivo de Tabla de Símbolos:}
    \begin{verbatim} 
CONTENIDOS DE LA TABLA # 1:
* LEXEMA : 'sql'
  Atributos:
--------- ---------
* LEXEMA : 'lt'
  Atributos:
--------- ---------
* LEXEMA : 'b'
  Atributos:
--------- ---------
* LEXEMA : 'dl'
  Atributos:
--------- ---------
    \end{verbatim}
\end{itemize}

\newpage
\subsection{Casos Erróneos}
\subsubsection{Caso Erróneo}
\begin{itemize}[label={-}, itemsep=-1em, topsep=0.5em, parsep=-0.2em]
    \item \textbf{Código:}
    \begin{verbatim} 
let int _ser = 45;
let float fl = 564.;
let boolean bl = true;
if (bl & false) int = 3;
    \end{verbatim}
    \item \textbf{Errores:}
    \begin{verbatim} 
Carácter ilegal '.' en línea 2 
Carácter ilegal '&' en línea 4
    \end{verbatim}
\end{itemize}

\subsubsection{Caso Erróneo}
\begin{itemize}[label={-}, itemsep=-1em, topsep=0.5em, parsep=-0.2em]
    \item \textbf{Código:}
    \begin{verbatim} 
let boolean xl3_ = false; / Doremifasollasido
let int del = 3;
del += 4;
if (xl3 == false) write('A'),
    \end{verbatim}
    \item \textbf{Error:}
    \begin{verbatim} 
    Carácter ilegal '/' en línea 1
    \end{verbatim}
\end{itemize}

\subsubsection{Caso Erróneo}
\begin{itemize}[label={-}, itemsep=-1em, topsep=0.0em, parsep=-0.2em]
    \item \textbf{Código:}
    \begin{verbatim} 
let int 3m = 56;
let float ad&d = 3.4;
if (ad&d < 3m) write('Error);
return ad&d;
    \end{verbatim}
    \item \textbf{Errores:}
    \begin{verbatim} 
Carácter ilegal '&' en línea 2
Carácter ilegal '&' en línea 3
Carácter ilegal "'" en línea 3
Carácter ilegal '&' en línea 4
    \end{verbatim}
\end{itemize}

\begin{landscape}
    \thispagestyle{empty}
    \subsection*{\centering Tabla LL1}
    \vfill
    \begin{table}[h!]
        \centering
        \noindent\hspace*{-0.065\textwidth}
        \resizebox{1.5\textwidth}{0.25\textheight}{%
            \begin{tabular}{|c|c|c|c|c|c|c|c|c|c|c|c|c|c|c|c|c|c|c|c|c|c|c|c|c|c|c|c|c|c|} \hline
                               & function                             & return           & write           & read      & if                     & else          & true                     & false                    & let                 & id                       & void & int             & float           & string                   & boolean         & intconst                 & realconst                & pluseq       & eq          & comma                      & semicolon               & oppar                    & clpar  & opbra      & clbra  & sum            & and           & $<$            & eof \\ \hline
                S              & LC S                                 & LC S             & LC S            & LC S      & LC S                   &               &                          &                          & LC S                & LC S                     &      &                 &                 &                          &                 &                          &                          &              &             &                            &                         &                          &        &            &        &                &               &                      & eof \\ \hline
                LC             &                                      & LS ;             & LS ;            & LS ;      & if(EX) CI & else CI &                          &                          & LS ;                & LS ;                     &      &                 &                 &                          &                 &                          &                          &              &             &                            &                         &                          &        &            &        &                &               &                      &     \\ \hline
                LF             & function TF id (AR)\{CU\} &                  &                 &           &                        &               &                          &                          &                     &                          &      &                 &                 &                          &                 &                          &                          &              &             &                            &                         &                          &        &            &        &                &               &                      &     \\ \hline
                CI       &                                      & LC               & LC              & LC        & LC                     & LC            &                          &                          & LC                  & LC                       &      &                 &                 &                          &                 &                          &                          &              &             &                            &                         &                          &        & \{CU\} &        &                &               &                      &     \\ \hline
                CU         &                                      & LC CU        & LC CU       & LC CU & LC CU              & LC CU     &                          &                          & LC CU           & LC CU                &      &                 &                 &                          &                 &                          &                          &              &             &                            &                         &                          &        &            & $\lambda$ &                &               &                      &     \\ \hline
                AR           &                                      &                  &                 &           &                        &               &                          &                          &                     &                          &      & TI id AM & TI id AM & TI id AM          & TI id AM &                          &                          &              &             &                            &                         &                          &        &            &        &                &               &                      &     \\ \hline
                AL    &                                      &                  &                 &           &                        &               & EX AML & EX AML &                     & EX AML &      &                 &                 & EX AML &                 & EX AML & EX AML &              &             &                            &                         & EX AML & $\lambda$ &            &        &                &               &                      &     \\ \hline
                AML &                                      &                  &                 &           &                        &               &                          &                          &                     &                          &      &                 &                 &                          &                 &                          &                          &              &             & , EX AML &                         &                          & $\lambda$ &            &        &                &               &                      &     \\ \hline
                AM        &                                      &                  &                 &           &                        &               &                          &                          &                     &                          &      &                 &                 &                          &                 &                          &                          &              &             & , TIpo id AM          &                         &                          & $\lambda$ &            &        &                &               &                      &     \\ \hline
                LS             &                                      & return ER & write EX & read id   &                        &               &                          &                          & let TI id AS & id IO                 &      &                 &                 &                          &                 &                          &                          &              &             &                            &                         &                          &        &            &        &                &               &                      &     \\ \hline
                IO          &                                      &                  &                 &           &                        &               &                          &                          &                     &                          &      &                 &                 &                          &                 &                          &                          & += EX & = EX &                            &                         & (AL)            &        &            &        &                &               &                      &     \\ \hline
                TF        &                                      &                  &                 &           &                        &               &                          &                          &                     &                          & void & TI            & TI            & TI                     & TI            &                          &                          &              &             &                            &                         &                          &        &            &        &                &               &                      &     \\ \hline
                TI           &                                      &                  &                 &           &                        &               &                          &                          &                     &                          &      & int             & float           & string                   & boolean         &                          &                          &              &             &                            &                         &                          &        &            &        &                &               &                      &     \\ \hline
                AS        &                                      &                  &                 &           &                        &               &                          &                          &                     &                          &      &                 &                 &                          &                 &                          &                          &              & = EX &                            & $\lambda$                  &                          &        &            &        &                &               &                      &     \\ \hline
                ER      &                                      &                  &                 &           &                        &               & EX                & EX                &                     & EX                &      &                 &                 & EX                &                 & EX                & EX                &              &             &                            & $\lambda$                  & EX                &        &            &        &                &               &                      &     \\ \hline
                EX      &                                      &                  &                 &           &                        &               & E1 EA  & E1 EA  &                     &                          &      &                 &                 & E1 EA  &                 & E1 EA  & E1 EA  &              &             &                            & E1 EA & E1 EA  &        &            &        &                &               &                      &     \\ \hline
                EA   &                                      &                  &                 &           &                        &               &                          &                          &                     &                          &      &                 &                 &                          &                 &                          &                          &              &             & $\lambda$                     & $\lambda$                  &                          & $\lambda$ &            &        &                & and EX &                      &     \\ \hline
                E1     &                                      &                  &                 &           &                        &               & E2 E1A & E2 E1A &                     & E2 E1A &      &                 &                 & E2 E1A &                 & E2 E1A & E2 E1A &              &             &                            &                         & E2 E1A &        &            &        &                &               &                      &     \\ \hline
                E1A  &                                      &                  &                 &           &                        &               &                          &                          &                     &                          &      &                 &                 &                          &                 &                          &                          &              &             & $\lambda$                     & $\lambda$                  &                          & $\lambda$ &            &        &                & $\lambda$        & $<$ E1 &     \\ \hline
                E2     &                                      &                  &                 &           &                        &               & E3 E2A & E3 E2A &                     & E3 E2A &      &                 &                 & E3 E2A &                 & E3 E2A & E3 E2A &              &             &                            &                         & E3 E2A &        &            &        &                &               &                      &     \\ \hline
                E2A  &                                      &                  &                 &           &                        &               &                          &                          &                     &                          &      &                 &                 &                          &                 &                          &                          &              &             & $\lambda$                     & $\lambda$                  &                          & $\lambda$ &            &        & sum E2 & $\lambda$        & $\lambda$               &     \\ \hline
                E3     &                                      &                  &                 &           &                        &               & true                     & false                    &                     & id E4            &      &                 &                 & string                   &                 & intconst                 & realconst                &              &             &                            &                         & (EX)              &        &            &        &                &               &                      &     \\ \hline
                E4     &                                      &                  &                 &           &                        &               &                          &                          &                     &                          &      &                 &                 &                          &                 &                          &                          &              &             & $\lambda$                     & $\lambda$                  & (AL)            & $\lambda$ &            &        & $\lambda$         & $\lambda$        & $\lambda$               &     \\ \hline
            \end{tabular}%
            \label{sec:tabla}
            }
    \end{table}
    \vfill
\end{landscape}

\begin{figure}[h!]
\centering
\includegraphics[width=0.45\textwidth]{../figures/goose.png} 
\caption{Barnacla canadiense intencional.}
\end{figure}

\end{document}
%%% Local Variables: 
%%% mode: latex
%%% TeX-master: t
%%% End: 
